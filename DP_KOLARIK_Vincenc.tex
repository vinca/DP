% options:
% thesis=B bachelor's thesis
% thesis=M master's thesis
% czech thesis in Czech language
% english thesis in English language
% hidelinks remove colour boxes around hyperlinks

\documentclass[thesis=M,english,hidelinks]{FITthesis}[2012/10/20]

\usepackage[utf8]{inputenc} % LaTeX source encoded as UTF-8
\usepackage{dirtree} %directory tree visualisation
\usepackage{graphicx} %graphics files inclusion
% \usepackage{subfig} %subfigures
% \usepackage{amsmath} %advanced maths
% \usepackage{amssymb} %additional math symbols
\usepackage{todonotes}
\usepackage[acronym,nonumberlist,toc,numberedsection=autolabel,nopostdot]{glossaries}

% % % % % % % % % % % % % % % % % % % % % % % % % % % % % % 
% FORMAL STUFF
% % % % % % % % % % % % % % % % % % % % % % % % % % % % % % 
\department{Department of Software Engineering}
\title{Applying the Normalized Systems Theory on Microservice Architecture}
\authorGN{Vincenc} %author's given name/names
\authorFN{Kolařík} %author's surname
\author{Vincenc Kolařík} %author's name without academic degrees
\authorWithDegrees{Bc. Vincenc Kolařík} %author's name with academic degrees
\supervisor{Ing. Robert Pergl, Ph.D.}
\placeForDeclarationOfAuthenticity{Prague}
\declarationOfAuthenticityOption{1} %select as appropriate, according to the desired license (integer 1-6)

% % % % % % % % % % % % % % % % % % % % % % % % % % % % % % 
% ABSTRACT & KEYWORDS
% % % % % % % % % % % % % % % % % % % % % % % % % % % % % % 

\acknowledgements{THANKS (remove entirely in case you do not with to thank anyone)}
\abstractEN{Summarize the contents and contribution of your work in a few sentences in English language.}
\abstractCS{V n{\v e}kolika v{\v e}t{\' a}ch shr{\v n}te obsah a p{\v r}{\' i}nos t{\' e}to pr{\' a}ce v {\v c}esk{\' e}m jazyce.}

\keywordsCS{Replace with comma-separated list of keywords in Czech.}
\keywordsEN{Replace with comma-separated list of keywords in English.}

% % % % % % % % % % % % % % % % % % % % % % % % % % % % % % 
% ACRONYMS
% % % % % % % % % % % % % % % % % % % % % % % % % % % % % % 
\makeglossaries
\newacronym{CVUT}{{\v C}VUT}{{\v C}esk{\' e} vysok{\' e} u{\v c}en{\' i} technick{\' e} v Praze}
\newacronym{FIT}{FIT}{Fakulta informa{\v c}n{\' i}ch technologi{\' i}}

\newacronym{SW}{SW}{software}
\newacronym{BE}{BE}{back-end}
\newacronym{FE}{FE}{front-end}

\newacronym{MS}{MS}{microservice}
\newacronym{NS}{NS}{Normalized Systems theory}

\newacronym{REST}{REST}{representational state transfer}
\newacronym{RPC}{RPC}{remote procedure call}
\newacronym{MAPE}{MAPE}{monitoring, analysis, planning, execution}

\newacronym{EA}{EA}{enterprise architecture}
\newacronym{MSA}{MSA}{microservice architecture}
\newacronym{SOA}{SOA}{service-oriented architecture}

\newacronym{AWS}{AWS}{Amazon Web Services}

%%%%%%%%%%%%%%%%%%%%%%%%%%%%%%%%%%%%%%%%%%%%%%%%%%%%%%%%%%%%
%%%%%%%%%%%%%%%%%%%%%%%%%%%%%%%%%%%%%%%%%%%%%%%%%%%%%%%%%%%%
%%                                                        %%
%%                      THE DOCUMENT                      %%
%%                                                        %%
%%%%%%%%%%%%%%%%%%%%%%%%%%%%%%%%%%%%%%%%%%%%%%%%%%%%%%%%%%%%
%%%%%%%%%%%%%%%%%%%%%%%%%%%%%%%%%%%%%%%%%%%%%%%%%%%%%%%%%%%%
\begin{document}

% % % % % % % % % % % % % % % % % % % % % % % % % % % % % % 
% 
% INTRODUCTION
% 
% % % % % % % % % % % % % % % % % % % % % % % % % % % % % % 
\setsecnumdepth{part}
\chapter{Introduction}
\todo[inline,size=\Large]{Missing whole chapter}
\begin{verbatim}
motivace

- same goals, different approach
- microservices = craftmanship
- NS theory of systems
- distributed 


- modularity, granularity, evolvability
- evolvability, adaptation to change
    
    
    
\end{verbatim}

Widely used throughout the industry by leading companies \cite{ms-who-is-using}.

\setsecnumdepth{all}

% % % % % % % % % % % % % % % % % % % % % % % % % % % % % % 
% 
% GOALS AND APPROACH
% 
% % % % % % % % % % % % % % % % % % % % % % % % % % % % % % 
\chapter{Goals and Approach}
\section{Goals}
In accordance to~thesis assignment the~following goals were set:
\begin{itemize}
	\item Analyze the current state-of-the-art design methods and industrial applications of microservice architecture.
	\item Identify key features and the most significant challenges encountered in~the~industry.
	\item Discuss compliance of the currently used methods with \acrlong{NS} and explore possibilities for improvements using \acrshort{NS}. 
	\item Formulate guidelines for designing microservices based on the results from the previous chapter, discuss them and demonstrate them on~a~case study if reasonable.
\end{itemize}

\section{Approach}
Microservices architecture is a \acrfull{EA} pattern. \acrshort{EA} could be analyzed in enormously broad context, spanning from business-IT alignment, through \acrshort{SW} development methodologies to choice of programming languages. This section discusses boundaries for this thesis to make the topic reasonably narrow without overlooking the quintessence of \acrlong{MSA}.

\subsection{Organizational Aspects}
Microservices are often valued for their positive effects on organizations and teams of engineers creating them. Development of end-to-end features and operational responsibility fosters DevOps culture \cite{devops-what-is}. Code ownership and cross-functional teams cultivate team spirit and nurture motivation of developers \cite{ms-fow-new-term-def, ms-modelling-with-petter, ms-building-ms}.

There's also a sociological observation called \textit{Conway's law} that states:
\begin{quote}
Any organization that designs a system (defined broadly) will produce a design whose structure is a copy of the organization's communication structure. \cite{conways-law}
\end{quote}

The bi-directional influence between software architecture and the organization that creates it is a remarkable topic and it would be unwise to ignore it while running a business. Despite that, it will be ignored in this thesis, as it would broaden its scope excessively. The \acrshort{NS} literature \cite{ns-recreating, ns-toward-general-theory} has the same attitude and avoids discussion how the organization influences the \acrshort{SW} and vice versa.

\subsection{Performance aspects}
The \acrlong{NS} describes a set of laws which, if applied strictly, guarantee a system to be free from combinatorial effects, i.e., to be indefinitely evolvable. Until now, the industrial \acrlong{SW} projects \cite{ns-it-isnt-different, ns-exploring-defence} are focused on the evolvability of the \acrshort{SW} artifact and doesn't value any other non-functional requirement as much as evolvability.

On the other hand, \acrlong{MS} came into being due to the need for performance and scalability \cite{ms-building-ms, ms-evolutionary-arch}. The evolvability aspect is never more important that those two mentioned. Therefore this thesis will respect this order and will not sacrifice performance and scalability for the sake of evolvability.

\subsection{Libraries, frameworks and other 3rd party technologies}
\begin{verbatim}
    - turbulent times, new frameworks emerging and be abandoned every day
    - https://github.com/mfornos/awesome-microservices not an extensive list of tech, but will suffice
    - don't bother with choosing and comparing particular 3rd party tech, too small detail
    - exctract the principcle and compare that
\end{verbatim}


\subsection{General Rules}
In an~effort to~make this work conceptually coherent, there were defined essential principles with which this thesis is created:
\begin{enumerate}
    \item Avoid discussion how organization influences the \acrshort{SW} and vice versa.
    \item Do not sacrifice performance and scalability for the sake of evolvability.
    \item When recommending to use an existing technology, extract the principle and recommend that 
\end{enumerate}

\section{Thesis Structure and Tasks}
To fulfill the goals of this thesis the following finer-grain task was defined. The~final structure of~this work is derived from this task list.
\begin{enumerate}
	\item Provide an~overview of~utilized theories and concepts (Chapter~\textit{\nameref{sec:theoretical_background}})
	\begin{itemize}
		\item Introduce key concepts of~\acrlong{NS}
		\item Describe the key principles and motivation for \acrlong{MSA}
	\end{itemize}

	\item Analyze the \acrlong{MSA} (Chapter~\textit{\nameref{sec:msa_analysis}})
	\begin{itemize}
		\item Perform a literature review on \acrshort{MS} and on application of \acrshort{NS} to \acrshort{MS} and/or related architecture styles and patterns
        \item Extract the essential concepts of state-of-the-art design of \acrshort{MS}
		\item Identify key concerns in \acrshort{MS} design and implementation
	\end{itemize}

	\item Examine compliance of \acrlong{MSA} to \acrlong{NS} (Chapter~\textit{\nameref{sec:msa_compliance}})
	\begin{itemize}
		\item Apply the \textit{Design Theorems for Stable Software} to the essential concepts from previous chapter
		\item Discuss 
	\end{itemize}	

	\item Summarize design guidelines for \acrshort{MSA} (Chapter~\textit{\nameref{sec:guidelines}})
	\begin{itemize}
		\item Provide concise and comprehensive overview of formulated guidelines
		\item Demonstrate guidelines on suitable case study
	\end{itemize}
	
	\item Summarize successes and failures (Chapter~\textit{\nameref{sec:conclusion}})
\end{enumerate}

% % % % % % % % % % % % % % % % % % % % % % % % % % % % % % 
% 
% NS INTRODUCTION
% 
% % % % % % % % % % % % % % % % % % % % % % % % % % % % % % 
\chapter{Introduction to Normalized Systems}
\label{sec:theoretical_background}
\section{Motivation for Constant Change}
\todo[inline,size=\Large]{Motivation for constant change}
\begin{verbatim}
    Agile
    Scrum
    Lean
    Learn build measure cycle
    
    Often used by companies in internet business which is the most volatile 
\end{verbatim}

\section{Normalized Systems Theory in Applied to Software}
\todo[inline,size=\Large]{Motivation for constant change}
\begin{verbatim}
- prevzit a doplnit z BP
\end{verbatim}


\section{Microservice Architecture}
\todo[inline,size=\Large]{Motivation for constant change}
\begin{verbatim}
 -- co sem vlastne napsat? vynechat uplne?
\end{verbatim}
\textcolor{magenta}{
microservices are small components, built around business capabilities [3], that are easy to understand, deploy, and scale independently, even using different technology stacks
}

\textcolor{magenta}{SOA is built on the concept of foster reuse: a share-as-much-as-possible architecture style, whereas microservices architecture is built on the concept of a share-as-little-as- possible style [8]. Given that service reuse has often been less than expected [9], instead of reusing existing microservices for new tasks or use cases, they should be “micro” enough to allow rapidly developing a new one that can coexist, evolve or replace the previous one according to the business needs [10].}

% % % % % % % % % % % % % % % % % % % % % % % % % % % % % % 
% 
% MICROSERVICE ANALYSIS
% 
% % % % % % % % % % % % % % % % % % % % % % % % % % % % % % 

\chapter{Analysis of the Microservice Architecture}
\label{sec:msa_analysis}

\section{Microservice scope}
\begin{verbatim}
    - domain driven development
    - focus on one thing and do it well
    
\end{verbatim}

\section{Inner vs. Outer Architecture}
\begin{verbatim}
    - focus on outer arch
    - inner - two arguments - it's sth. that can be rewritten by a team in two weeks
    - mention other definition
\end{verbatim}
\begin{figure}
 \caption{\textit{Inner vs. Outer Architecture}}
 \label{fig:in-vs-out-arch}
 \missingfigure{\nameref{fig:in-vs-out-arch}}
\end{figure}


\section{Inter-microservice Communication}
\begin{verbatim}
    - direct vs. messaging
    - smart endpoints vs. dumb pipes
\end{verbatim}

\subsection{Interaction Models}
Interaction Models refers to the communication flow among components. Possible values: synchronous, asynchronous.

\subsection{Service Interfaces}
\textcolor{magenta}{Service Interfaces are the different means of specifying contracts (if any) for the communication of microservices [45]. Possible values: formal (defined through a formal contract), tech-tied (the interface is tied to the implemen- tation technology), ad-hoc (defined in a novel language).}


\subsection{Data Exchange Protocol}
Data Exchange are the protocols used to represent the communication. Possible values: REST/HTTP, RPC-alike, message queues, other.



\section{User Interface}
\begin{verbatim}
    - as seen on figure \nameref{fig:in-vs-out-arch} there's plenty of devices that can serve as frontend
    - tenths of operation systems, tenths of web browsers,  
\end{verbatim}

\section{Persistence}

\textcolor{magenta}{Eventual Consistency: Maintaining strong consistency is extremely difficult for a distributed system, which means everyone has to manage eventual consistency.}

\textcolor{magenta}{Data Storage usually integrated multiple services in legacy systems, but in microservices architectures it is mandatory to find seams in the databases and use the right technologies to split them out cleanly \cite{ms-building-ms}. Possible values: SQL, graph-oriented, document-oriented, other.}

\section{Deployment \& Operations}
\textcolor{magenta}{Deployment encompasses how and where services are actually hosted and deployed. Although the cloud has been adopted as the de-facto platform for microservices [44], there are several alternatives into and out of the cloud.}

\textcolor{magenta}{– Platform can be customized due to privacy, security or business constraints. Possible values: public cloud, private cloud, in-house.}

\textcolor{magenta}{– Management encompasses the responsive reaction to failures and changing environmental conditions, minimizing human intervention [7]. Possible val- ues: built-in cloud services (e.g., AWS Cloudwatch and Autoscaling), third- party services (e.g., Rightscale, New Relic), ad-hoc solutions (i.e., tied to the particular approach).}

\subsection{Scalability / Elasticity}
– Scalability and elasticity refer to the capability to rapidly adjust the overall capacity of the platform by adding or removing resources, also minimizing human intervention [7]. Possible values: vendor-provided, autonomic MAPE loops, configuration servers, other.
\subsection{Availability and Resilience}
\textcolor{magenta}{Simply handling both service-level and low-level failures that demand for persistence and recovery techniques [45]. Possible values: resilience patterns, fault injection, error-handling policies, resilience tests, other.}

\section{Security}
\begin{verbatim}
    - Sessions auth
    - JWT/token authentication
\end{verbatim}

\missingfigure{JWT vs. Session authentication call diagram}

\subsection{Observability}
\subsubsection{Monitoring}
\subsubsection{Logging}


% % % % % % % % % % % % % % % % % % % % % % % % % % % % % % 
% 
% TOWARDS STABLE MICROSERVICES
% 
% % % % % % % % % % % % % % % % % % % % % % % % % % % % % % 

\chapter{Towards Stable Microservice Architecture}
\label{sec:msa_compliance}
\section{Mapping NS Elements to Microservice Architecture}

\todo[inline,size=\Large]{Missing whole chapter}

\chapter{Guidelines for Stable Microservice Architecture}
\label{sec:guidelines}
\todo[inline,size=\Large]{Missing whole chapter}

% % % % % % % % % % % % % % % % % % % % % % % % % % % % % % 
% 
% CONCLUSION
% 
% % % % % % % % % % % % % % % % % % % % % % % % % % % % % % 
\setsecnumdepth{part}
\chapter{Conclusion}
\label{sec:conclusion}
\section{Evaluation of Goals}
\todo[inline,size=\Large]{Missing section}

\section{Thoughts on Utilization}
\todo[inline,size=\Large]{Missing section}

\section{Future Work}
\todo[inline,size=\Large]{Missing section}

%%%%%%%%%%%%%%%%%%%%%%%%%%%%%%%%%%%%%%%%%%%%%%%%%%%%%%%%%%%%
%%%%%%%%%%%%%%%%%%%%%%%%%%%%%%%%%%%%%%%%%%%%%%%%%%%%%%%%%%%%
%%                                                        %%
%%                    REFERENCES ETC.                     %%
%%                                                        %%
%%%%%%%%%%%%%%%%%%%%%%%%%%%%%%%%%%%%%%%%%%%%%%%%%%%%%%%%%%%%
%%%%%%%%%%%%%%%%%%%%%%%%%%%%%%%%%%%%%%%%%%%%%%%%%%%%%%%%%%%%

\bibliographystyle{iso690}
\bibliography{DP_references}

\setsecnumdepth{all}
\appendix

\printglossary[type=\acronymtype,toctitle=]

\chapter{Contents of enclosed CD}
\begin{figure}
	\dirtree{%
		.1 readme.txt\DTcomment{the file with CD contents description}.
        .1 Figures\DTcomment{source files of figures used in the thesis}.
        .1 Text\DTcomment{thesis text}.
            .2 DP\_Kolarik\_Vincenc.pdf\DTcomment{PDF version of~the~thesis}.
            .2 src\DTcomment{\LaTeX{} source codes of~the~thesis}.
    }
\end{figure}

\listoftodos

\end{document}
