% options:
% thesis=B bachelor's thesis
% thesis=M master's thesis
% czech thesis in Czech language
% english thesis in English language
% hidelinks remove colour boxes around hyperlinks

\documentclass[thesis=M,english,hidelinks]{FITthesis}[2012/10/20]

\usepackage[utf8]{inputenc} % LaTeX source encoded as UTF-8
\usepackage{dirtree} %directory tree visualisation
\usepackage{graphicx} %graphics files inclusion
% \usepackage{subfig} %subfigures
% \usepackage{amsmath} %advanced maths
% \usepackage{amssymb} %additional math symbols
\usepackage{todonotes}
\usepackage[acronym,nonumberlist,toc,numberedsection=autolabel,nopostdot]{glossaries}

% % % % % % % % % % % % % % % % % % % % % % % % % % % % % % 
% FORMAL STUFF
% % % % % % % % % % % % % % % % % % % % % % % % % % % % % % 
\department{Department of Software Engineering}
\title{Applying the Normalized Systems Theory on Microservice Architecture}
\authorGN{Vincenc} %author's given name/names
\authorFN{Kolařík} %author's surname
\author{Vincenc Kolařík} %author's name without academic degrees
\authorWithDegrees{Bc. Vincenc Kolařík} %author's name with academic degrees
\supervisor{Ing. Robert Pergl, Ph.D.}
\placeForDeclarationOfAuthenticity{Prague}
\declarationOfAuthenticityOption{1} %select as appropriate, according to the desired license (integer 1-6)

% % % % % % % % % % % % % % % % % % % % % % % % % % % % % % 
% ABSTRACT & KEYWORDS
% % % % % % % % % % % % % % % % % % % % % % % % % % % % % % 

\acknowledgements{THANKS (remove entirely in case you do not with to thank anyone)}
\abstractEN{Summarize the contents and contribution of your work in a few sentences in English language.}
\abstractCS{V n{\v e}kolika v{\v e}t{\' a}ch shr{\v n}te obsah a p{\v r}{\' i}nos t{\' e}to pr{\' a}ce v {\v c}esk{\' e}m jazyce.}

\keywordsCS{Replace with comma-separated list of keywords in Czech.}
\keywordsEN{Replace with comma-separated list of keywords in English.}

% % % % % % % % % % % % % % % % % % % % % % % % % % % % % % 
% ACRONYMS
% % % % % % % % % % % % % % % % % % % % % % % % % % % % % % 
\makeglossaries
\newacronym{CVUT}{{\v C}VUT}{{\v C}esk{\' e} vysok{\' e} u{\v c}en{\' i} technick{\' e} v Praze}
\newacronym{FIT}{FIT}{Fakulta informa{\v c}n{\' i}ch technologi{\' i}}
\newacronym{NS}{NS}{Normalized Systems theory}
\newacronym{MS}{MS}{Microservice}
\newacronym{MSA}{MSA}{Microservice Architecture}
\newacronym{SOA}{SOA}{Service-oriented Architecture}

% % % % % % % % % % % % % % % % % % % % % % % % % % % % % % 
% CONTENT
% % % % % % % % % % % % % % % % % % % % % % % % % % % % % % 
\begin{document}
\listoftodos 

%
% GOALS AND APPROACH
%
\setsecnumdepth{part}
\chapter{Introduction}
\todo{whole chapter}
\begin{verbatim}
motivace

- same goals, different approach
- microservices = craftmanship
- NS theory of systems
- distributed 
    
\end{verbatim}

Widely used throughout the industry by leading companies \cite{ms-who-is-using}.

\setsecnumdepth{all}

%
% GOALS AND APPROACH
%

\chapter{Goals and Approach}
\section{Goals}
In accordance to~thesis assignment the~following goals were set:
\begin{itemize}
	\item Analyze the current state-of-the-art design methods and industrial applications of microservice architecture.
	\item Identify key features and the most significant challenges encountered in~the~industry.
	\item Discuss compliance of the currently used methods with \acrlong{NS} and explore possibilities for improvements using \acrshort{NS}. 
	\item Formulate guidelines for designing microservices based on the results from the previous chapter, discuss them and demonstrate them on~a~case study if reasonable.
\end{itemize}

\section{Approach}
\subsection{General Rules}
In an~effort to~make this work conceptually coherent, there were defined essential principles with which this thesis is created:
\todo{General rules - design principles (napsat nebo smazat)}
\begin{itemize}
    \item don't ignore implementation details
    
% 	\item Primary aim is not to~create something useful, but to~explore possibilities and map the~state of~the~art.
% 	\item Energy will be devoted to~be theoretically correct, complete and compliant with both instruments.
% 	\item The~method has been designed without direct access to~\acrshort{nse}, so the~user experience cannot be simulated. Therefore its considered as secondary aspect.
% 	\item Flexibility will be the~major requirement for both theoretical and practical parts.
% 	\item Implemented prototyped should serve as proof of~concept, not a~production grade tool.
% 	\item Theoretical method as well as prototype tool will presume working with valid data models and will posses minimal or none validation mechanisms.
% 	\item Transformation method will endeavour to~avoid any changes in~design of~the~model. So the~model, created without any intention of~using this tool, can be transformed as-is or with minimal modifications. \emph{Model design guidelines} will be created to~cover those inevitable design rules (e.g. using only data types supported by the~\acrshort{nse}).
\end{itemize}

\section{Thesis Structure and Tasks}
To fulfill the goals of this thesis the following finer-grain task was defined. The~final structure of~this work is derived from this task list.
\begin{enumerate}
	\item Provide an~overview of~utilized theories and concepts (Chapter~\textit{\nameref{sec:theoretical_background}})
	\begin{itemize}
		\item Introduce key concepts of~\acrlong{NS}
		\item Describe the key principles and motivation for \acrlong{MSA}
	\end{itemize}

	\item Analyze the \acrlong{MSA} (Chapter~\textit{\nameref{sec:msa_analysis}})
	\begin{itemize}
		\item Perform a literature review on \acrshort{MS} and on application of \acrshort{NS} to \acrshort{MS} and/or related architecture styles and patterns
        \item Extract the essential concepts of state-of-the-art design of \acrshort{MS}
		\item Identify key concerns in \acrshort{MS} design and implementation
	\end{itemize}

	\item Examine compliance of \acrlong{MSA} to \acrlong{NS} (Chapter~\textit{\nameref{sec:msa_compliance}})
	\begin{itemize}
		\item Apply the \textit{Design Theorems for Stable Software} to the essential concepts from previous chapter
		\item Discuss 
	\end{itemize}	

	\item Summarize design guidelines for \acrshort{MSA} (Chapter~\textit{\nameref{sec:guidelines}})
	\begin{itemize}
		\item Provide concise and comprehensive overview of formulated guidelines
		\item Demonstrate guidelines on suitable case study
	\end{itemize}
	
	\item Summarize successes and failures (Chapter~\textit{\nameref{sec:conclusion}})
\end{enumerate}

\chapter{Theoretical Background}
\label{sec:theoretical_background}
\todo{whole chapter}
something about \cite{ns-towards-evolvable}, or \cite{ns-recreating} and for sure \cite{ns-toward-general-theory}
also something about \cite{ms-vs-soa}, or \cite{ms-who-is-using} and \cite{ms-goto-challenges-george}, and \cite{ms-building-ms} and \cite{ms-evolutionary-arch}
\cite{ms-design-tradeoffs}
\cite{ms-underestimated-network-impact}
\cite{ms-integrating-with-adaptable-ea}

\chapter{Analysis of the Microservice Architecture}
\label{sec:msa_analysis}
\todo{whole chapter}

\chapter{Towards Stable Microservice Architecture}
\label{sec:msa_compliance}
\todo{whole chapter}

\chapter{Guidelines for Stable Microservice Architecture}
\label{sec:guidelines}
\todo{whole chapter}

\setsecnumdepth{part}
\chapter{Conclusion}
\label{sec:conclusion}
\todo{whole chapter}

% % % % % % % % % % % % % % % % % % % % % % % % % % % % % % 
% REFERENCES ETC.
% % % % % % % % % % % % % % % % % % % % % % % % % % % % % % 

\bibliographystyle{iso690}
\bibliography{DP_references}

\setsecnumdepth{all}
\appendix

\printglossary[type=\acronymtype,toctitle=]

\chapter{Contents of enclosed CD}
\begin{figure}
	\dirtree{%
		.1 readme.txt\DTcomment{the file with CD contents description}.
        .1 Figures\DTcomment{source files of figures used in the thesis}.
        .1 Text\DTcomment{thesis text}.
            .2 DP\_Kolarik\_Vincenc.pdf\DTcomment{PDF version of~the~thesis}.
            .2 src\DTcomment{\LaTeX{} source codes of~the~thesis}.
    }
\end{figure}

\end{document}
